\section{Valutazione orari e costi}
Per la corretta esecuzione di un progetto il totale delle ore produttive deve essere suddiviso tra più figure professionali, ciascuna con un ruolo, un costo ed un ammontare di ore diverso. Inoltre all'interno del gruppo farmacode ogni studente si impegna a coprire in modo equo ciascuno di questi ruoli in rotazione. \\
Di seguito sono riportate le tabelle con la suddivisione delle ore per ciascun ruolo interno e per ciascun membro del gruppo. Il totale delle ore produttive per componente ammonta a 95 ore, per un totale di 665 ore produttive totali del gruppo.
    
    \subsection{Ruoli interni al progetto con relativi costi ed orari}
        \setlength{\arrayrulewidth}{0.5mm}
        \setlength{\tabcolsep}{20pt}
        \renewcommand{\arraystretch}{2}
        \rowcolors{2}{gray!12}{gray!35}
        \begin{tabular}[*{10}{t}]{|c|p{4.7cm}*{2}{|c}|}
            \hline
            Ruolo & Responsabilità & Costo Orario & Ore per Ruolo \\
            \hline
            Responsabile & Coordina l’elaborazione di piani e scadenze, approva il rilascio di prodotti parziali o finali (SW, documenti), coordina le attività del gruppo. & 30 & 70 \\
            Amministratore & Assicura l’efficienza di procedure, strumenti e tecnologie a supporto del way of working. & 20 & 35 \\
            Analista &  Svolge le attività di analisi dei requisiti. & 25 & 35 \\
            Progettista & Svolge le attività di progettazione (design). & 25 & 133 \\
            Programmatore & Svolge le attività di codifica. & 15 & 252 \\
            Verificatore & Svolge le attività di verifica. & 15 & 140 \\
            \hline
             & & Totale Costo & Totale Ore \\
             & & 12880 & 665 \\
            \hline
        \end{tabular}

    \subsection{Partizione oraria dei ruoli}
    A ciascun ruolo è stato assegnato un ammontare dividendo il totale di ore previsto in base alle seguenti considerazioni:
    \begin{itemize}
        \item I ruoli presenti si possono dividere in una categoria "gestionale" che comprende i ruoli di Responsabile, Amministratore e Analista ed in una categoria "pratica" che comprende i ruoli di Progettista, Programmatore e Verificatore. La seconda svolgerà la maggior parte delle ore lavorative;
        \item All'interno della categoria "gestionale" al Responsabile spettano più compiti e deve gestire tutti i membri del gruppo di entrambe le categorie, ci si aspetta dunque una mole oraria doppia rispetto agli altri ruoli di questa categoria;
        \item All'interno della categoria "pratica" la maggior parte del lavoro verrà effettuata dal programmatore, figura che sarà inoltre presente in coppia, dunque ci si aspetta anche qui una mole oraria doppia rispetto agli altri ruoli di questa categoria, questo anche perché il programmatore dovrà sistemare o modificare il codice una volta confrontatosi con Progettista e Verificatore a seguito di eventuali problemi.
    \end{itemize}
        
    \subsection{Componenti del gruppo con relativa suddivisione degli orari in base al ruolo}
        \setlength{\arrayrulewidth}{0.5mm}
        \setlength{\tabcolsep}{20pt}
        \renewcommand{\arraystretch}{2}
        \rowcolors{2}{gray!12}{gray!35}
        \begin{tabular}[*{8}{t}]{*{8}{|c}|}
            \hline
            Membro & R & A1 & A2 & P1 & P2 & V & TOT \\
            \hline
            Baggio & 10 & 5 & 5 & 19 & 36 & 20 & 95 \\
            Bomben & 10 & 5 & 5 & 19 & 36 & 20 & 95 \\
            Carraro & 10 & 5 & 5 & 19 & 36 & 20 & 95 \\
            Favaron & 10 & 5 & 5 & 19 & 36 & 20 & 95 \\
            Pandolfo & 10 & 5 & 5 & 19 & 36 & 20 & 95 \\
            Passarella & 10 & 5 & 5 & 19 & 36 & 20 & 95 \\
            Rosson & 10 & 5 & 5 & 19 & 36 & 20 & 95 \\
            \hline
        \end{tabular}} 
        \vspace{0.5cm} 
   
        Ciascuna delle sigle nella tabella si riferisce ad uno specifico ruolo:
        \begin{itemize}
            \item R: Responsabile;
            \item A1: Amministratore;
            \item A2: Analista;
            \item P1: Progettista;
            \item P2: Programmatore;
            \item V: Verificatore.
        \end{itemize}
    
    \subsection{Partizione interna dei ruoli}
    Ciascun ammontare di ore per ogni ruolo è stato ripartito equamante tra ogni componente del gruppo farmacode. I ruoli nel corso del progetto verranno continuamente mutati in rotazione tra i membri del gruppo, fino al terminare le ore produttive di ciascun ruolo per tutti i componenti del gruppo.

\section{Preventivo dei costi}
    In base al totale delle ore produttive richieste dal progetto ed alla loro suddivisione nei vari ruoli, i quali prevedono costi orari, si prevede un costo finale del progetto relativo alle ore produttive che ammonta a 12.880€ netti. Costo verificabile tramite la prima tabella mostrata nella presente.

\section{Scadenza prevista}
    La scadenza stimata di consegna del prodotto terminato relativo al secondo capitolato "Sistemi di raccomandazione" dell'azienda "Ergon Informatica" si loca in data 25 Marzo 2024.




