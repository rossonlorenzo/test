\section{Processi di supporto}
\subsection{Documentazione}

\subsubsection{Descrizione e scopo} %da cambiare qua, esprimere che il glossario è su un altro documento

La documentazione software è una raccolta di informazioni che mirano a descrivere il prodotto che accompagnano, per coloro che lo sviluppano, distribuiscono e utilizzano. La sua principale finalità è facilitare il lavoro dei membri del team durante il processo di sviluppo, consentendo di tracciare e documentare in modo dettagliato tutte le attività e i processi coinvolti. Questo a semplificare la manutenzione del software e a migliorare la qualità del prodotto finale.

La documentazione software dovrebbe plausibilmente includere la definizione di regole chiare e concise per la redazione dei documenti. Inoltre, è importante stabilire una struttura uniforme e coerente per tutti i documenti che resti il più possibile invariata durante tutto il ciclo di vita del software, garantendo così una coesione complessiva.

\subsubsection{Caratterizzazione dei documenti}
\begin{itemize}
    \item Formali:
    Sono i documenti che andranno a formare La documentazione software del prodotto. In quanto tali sono sottoposti a versionamento e a processi di verifica e approvazione.
    Essi comprendono documenti interni, utili quindi ai membri del team di sviluppo, ed esterni, destinati a proponente e commitente.

    Complessivamente ne fanno parte:
    \begin{itemize}
        \item Interni: 
        \begin{itemize}
            \item Norme di progetto, rappresentano il "way of working";
            \item Verbali interni e esterni, a uso consultativo;
        \end{itemize}
        \item Esterni:
         \begin{itemize}
            \item Analisi dei Requisiti;
            \item Piano di Progetto; 
            \item Piano di Qualifica; 
            \item Glossario; 
            \item Verbali interni e esterni, attestanti di quanto discusso.
        \end{itemize}
    \end{itemize}
    \item Informali:
    Sono i documenti interni non destinati alla divulgazione con esterni e fini a loro stessi. Perciò non necessitano di versionamento.
    Spesso sono bozze in preparazione a documenti formali, o note e appunti generiche.
\end{itemize}
\subsubsection{Template e layout}
I documenti sono realizzati in \LaTeX\ utilizzando come editor. Il è standard per tutti i documenti, ad eccezione dei verbali.

\subsubsection{Struttura dei documenti}




\subsection{Riferimenti}
